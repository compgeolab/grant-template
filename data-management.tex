% Description of activities for the project team
%
%%%%%%%%%%%%%%%%%%%%%%%%%%%%%%%%%%%%%%%%%%%%%%%%%%%%%%%%%%%%%%%%%%%%%%%%%%%%%%%
% Set a class and general configuration
\documentclass[onecolumn,a4paper,12pt]{article}

%%%%%%%%%%%%%%%%%%%%%%%%%%%%%%%%%%%%%%%%%%%%%%%%%%%%%%%%%%%%%%%%%%%%%%%%%%%%%%%
% Set variables with the title, authors, etc.
% Edit the values and they will propagate to the document and head/foot

\newcommand{\Title}{%
  Grant proposal title in English
}
\newcommand{\TitlePt}{%
  Grant proposal title in Portuguese
}
\newcommand{\TitleShort}{Short title for the footer}
\newcommand{\Year}{2024}
\newcommand{\TeamShort}{Uieda \textit{et al.}}
\newcommand{\PIname}{Leonardo Uieda}
\newcommand{\PIORCID}{0000-0001-6123-9515}
\newcommand{\Institution}{%
  Instituto de Astronomia, Geofísica e Ciências Atmosféricas -- Universidade de
  São Paulo
}
\newcommand{\FundingCall}{FAPESP Jovem Pesquisador}

\newcommand{\DocumentTitle}{%
  Data management plan
}
\renewcommand{\TitleShort}{Data management plan}

%%%%%%%%%%%%%%%%%%%%%%%%%%%%%%%%%%%%%%%%%%%%%%%%%%%%%%%%%%%%%%%%%%%%%%%%%%%%%%%
% Import the required packages
\usepackage[utf8]{inputenc}
\usepackage[TU]{fontenc}
\usepackage[english]{babel}
\usepackage{amsmath}
\usepackage{amssymb}
\usepackage{hyperref}
\usepackage{fancyhdr}
\usepackage{geometry}
\usepackage{microtype}
\usepackage{siunitx}
\usepackage{xcolor}
% improved urls with proper hyphenation
\usepackage{xurl}
% To control the style of section titles
\usepackage{titlesec}
% Use a different font
\usepackage[scaled=0.9,sfdefault]{notomath}
% Control the font size
\usepackage{anyfontsize}
\usepackage{setspace}
% To get the number of pages in the document
\usepackage{lastpage}
\usepackage{lipsum}
\usepackage{ragged2e}
% To control hyphenation for individual blocks of text
\usepackage{hyphenat}
% To add line numbers
\usepackage{reledmac}

%%%%%%%%%%%%%%%%%%%%%%%%%%%%%%%%%%%%%%%%%%%%%%%%%%%%%%%%%%%%%%%%%%%%%%%%%%%%%%%
% Configuration of the document
\geometry{%
  left=30mm,
  right=15mm,
  top=15mm,
  bottom=15mm,
  headsep=0mm,
  headheight=0mm,
  footskip=7mm,
  includehead=true,
  includefoot=true
}

% Control line and table row spacing
\onehalfspacing
\renewcommand{\arraystretch}{1.5}

% Custom colors
\definecolor{darkgray}{gray}{0.4}
\definecolor{mediumgray}{gray}{0.5}
\definecolor{lightgray}{gray}{0.9}
\definecolor{mediumblue}{HTML}{2060c2}
\definecolor{lightblue}{HTML}{f7faff}

% Make urls use the same font as every other text
\urlstyle{same}

% Number every single line instead of every 5th line
\firstlinenum{1}
\linenumincrement{1}

% Configure hyperref and add PDF metadata
\hypersetup{
    colorlinks,
    allcolors=mediumblue,
    pdftitle={\DocumentTitle},
    pdfauthor={\PIname},
    breaklinks=true,
}

% Configure header and footer
% Inspired by LaPreprint: https://github.com/roaldarbol/LaPreprint
\newcommand{\Separator}{\hspace{3pt}|\hspace{3pt}}
\newcommand{\FooterFont}{\footnotesize\color{mediumgray}}
\pagestyle{fancy}
\fancyhf{}
\lfoot{%
  \FooterFont{}
  \TeamShort{} (\Year)
  \Separator{}
  \TitleShort
}
\rfoot{%
  \FooterFont{}
  \FundingCall{}
  \Separator{}
  \thepage\space of\space \pageref*{LastPage}
}
\renewcommand{\headrulewidth}{0pt}
\renewcommand{\footrulewidth}{1pt}
\preto{\footrule}{\color{lightgray}}
\fancypagestyle{plain}{%
  \fancyhf{}
  \lfoot{}
  \rfoot{}
  \renewcommand{\footrulewidth}{0pt}
}


\begin{document}


\begin{spacing}{2}
  \noindent
  {\LARGE \textbf{\DocumentTitle}}
\end{spacing}

\noindent
\textbf{Project:} \Title

%%%%%%%%%%%%%%%%%%%%%%%%%%%%%%%%%%%%%%%%%%%%%%%%%%%%%%%%%%%%%%%%%%%%%%%%%%%%%%%
% Main text - EDIT HERE

% Texto de até duas páginas, contendo as seguintes informações
% a) Descrição dos dados e metadados produzidos pelo projeto - por exemplo,
% amostras, registros de coleta, formulários, modelos, resultados experimentais,
% software, gráficos, mapas, vídeos, planilhas, gravações de áudio, bancos de
% dados, material didático e outros.
% b) Quando aplicável, restrições legais ou éticas para compartilhamento de tais
% dados, políticas para garantir a privacidade, confidencialidade, segurança,
% propriedade intelectual e outros.
% c) Política de preservação e compartilhamento (por exemplo, compartilhamento
% imediato ou apenas após a aceitação da publicação associada). Período de
% carência (antes do compartilhamento) e período durante o qual os dados serão
% preservados e disponibilizados.
% d) Descrição de mecanismos, formatos e padrões para armazenar tais itens de
% forma a torná-los acessíveis por terceiros. Esta descrição pode incluir o uso
% de repositórios e serviços de outras instituições.

\section{Data description}

\lipsum[1]

\section{Sharing conditions}

\lipsum[2]

\section{Preservation and sharing}

\lipsum[3-4]


\end{document}
%------------------------------------------------------------------------------


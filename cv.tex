% CV summary (súmula curricular)
%
%%%%%%%%%%%%%%%%%%%%%%%%%%%%%%%%%%%%%%%%%%%%%%%%%%%%%%%%%%%%%%%%%%%%%%%%%%%%%%%
% Set a class and general configuration
\documentclass[onecolumn,a4paper,11pt]{article}

%%%%%%%%%%%%%%%%%%%%%%%%%%%%%%%%%%%%%%%%%%%%%%%%%%%%%%%%%%%%%%%%%%%%%%%%%%%%%%%
% Set variables with the title, authors, etc.
% Edit the values and they will propagate to the document and head/foot

\newcommand{\Title}{%
  Grant proposal title in English
}
\newcommand{\TitlePt}{%
  Grant proposal title in Portuguese
}
\newcommand{\TitleShort}{Short title for the footer}
\newcommand{\Year}{2024}
\newcommand{\TeamShort}{Uieda \textit{et al.}}
\newcommand{\PIname}{Leonardo Uieda}
\newcommand{\PIORCID}{0000-0001-6123-9515}
\newcommand{\Institution}{%
  Instituto de Astronomia, Geofísica e Ciências Atmosféricas -- Universidade de
  São Paulo
}
\newcommand{\FundingCall}{FAPESP Jovem Pesquisador}

\newcommand{\Name}{Leonardo Uieda}
\newcommand{\DocumentTitle}{%
  Súmula curricular - \Name
}
\renewcommand{\TitleShort}{\DocumentTitle}

%%%%%%%%%%%%%%%%%%%%%%%%%%%%%%%%%%%%%%%%%%%%%%%%%%%%%%%%%%%%%%%%%%%%%%%%%%%%%%%
% Import the required packages
\usepackage[utf8]{inputenc}
\usepackage[TU]{fontenc}
\usepackage[english]{babel}
\usepackage{amsmath}
\usepackage{amssymb}
\usepackage{hyperref}
\usepackage{fancyhdr}
\usepackage{geometry}
\usepackage{microtype}
\usepackage{siunitx}
\usepackage{xcolor}
% Disable hyphenation
\usepackage[none]{hyphenat}
% Left-align instead of justify
\usepackage[document]{ragged2e}
% improved urls with proper hyphenation
\usepackage{xurl}
% To control the style of section titles
\usepackage{titlesec}
% Use a different font
\usepackage[scaled=0.9,sfdefault]{notomath}
\usepackage{academicons}
\usepackage{fontawesome5}
% Control the font size
\usepackage{anyfontsize}
\usepackage{setspace}
% Control spacing in enumerates
\usepackage{enumitem}
% To get the number of pages in the document
\usepackage{lastpage}
\usepackage{lipsum}
\usepackage{ragged2e}
% Configure section titles
\usepackage{titlesec}
% To control hyphenation for individual blocks of text
\usepackage{hyphenat}
% For fancy and multipage tables
\usepackage{tabularx}
\usepackage{ltablex}
% For new environments
\usepackage{environ}

%%%%%%%%%%%%%%%%%%%%%%%%%%%%%%%%%%%%%%%%%%%%%%%%%%%%%%%%%%%%%%%%%%%%%%%%%%%%%%%
% Configuration of the document
\geometry{%
  left=25mm,
  right=20mm,
  top=20mm,
  bottom=20mm,
  headsep=0mm,
  headheight=0mm,
  footskip=7mm,
  includehead=true,
  includefoot=true
}

% Control line and table row spacing
\renewcommand{\baselinestretch}{1}
\renewcommand{\arraystretch}{1.1}
% Remove space between items in itemize and enumerate
\setlist{nosep}

% No indentation
\setlength\parindent{0cm}

% Set the spacing and format of sections
\titleformat{\section}
  {\normalfont\Large\mdseries} % format
  {} % label
  {0pt} % separation (left separation for hang)
  {} % text before title
  [\titlerule] % text after title
\titlespacing*{\section}
  {0pt} % left pad
  {0.1cm} % before
  {0cm} % after

% Disable number of sections. Use this instead of "section*" so that the sections still
% appear as PDF bookmarks. Otherwise, would have to add the table of contents entries
% manually.
\makeatletter
\renewcommand{\@seccntformat}[1]{}
\makeatother

% Custom colors
\definecolor{darkgray}{gray}{0.4}
\definecolor{mediumgray}{gray}{0.5}
\definecolor{lightgray}{gray}{0.9}
\definecolor{mediumblue}{HTML}{2060c2}
\definecolor{lightblue}{HTML}{f7faff}

% Make urls use the same font as every other text
\urlstyle{same}

% Configure hyperref and add PDF metadata
\hypersetup{
    colorlinks,
    allcolors=mediumblue,
    pdftitle={\DocumentTitle},
    pdfauthor={\PIname},
    breaklinks=true,
}

% Configure header and footer
% Inspired by LaPreprint: https://github.com/roaldarbol/LaPreprint
\newcommand{\Separator}{\hspace{3pt}|\hspace{3pt}}
\newcommand{\FooterFont}{\footnotesize\color{mediumgray}}
\pagestyle{fancy}
\fancyhf{}
\lfoot{%
  \FooterFont{}
  \TeamShort{} (\Year)
  \Separator{}
  \TitleShort
}
\rfoot{%
  \FooterFont{}
  \FundingCall{}
  \Separator{}
  \thepage\space of\space \pageref*{LastPage}
}
\renewcommand{\headrulewidth}{0pt}
\renewcommand{\footrulewidth}{1pt}
\preto{\footrule}{\color{lightgray}}
\fancypagestyle{plain}{%
  \fancyhf{}
  \lfoot{}
  \rfoot{}
  \renewcommand{\footrulewidth}{0pt}
}

% Define a new environment to place all CV entries in a 2-column table.
% Left column are the dates, right column the entries.
\newcommand{\TablePad}{\vspace{-0.2cm}}
\NewEnviron{EntriesTableDuration}{
\TablePad
\begin{tabularx}{\textwidth}{@{}p{0.10\textwidth}@{\hspace{0.02\textwidth}}p{0.88\textwidth}@{}}
  \BODY
\end{tabularx}
\TablePad
}
\newcommand{\Duration}[2]{\fontsize{9pt}{0}\selectfont #1 - #2}
\newcommand{\DOI}[1]{doi:\href{https://doi.org/#1}{#1}}
\newcommand{\Website}[1]{\href{https://#1}{#1}}



\begin{document}

{\LARGE \textbf{\DocumentTitle}}

%%%%%%%%%%%%%%%%%%%%%%%%%%%%%%%%%%%%%%%%%%%%%%%%%%%%%%%%%%%%%%%%%%%%%%%%%%%%%%%
% Main text - EDIT HERE

\section{Formação}

\begin{EntriesTableDuration}
  \Duration{2004}{2009}  &
  \textbf{Bacharelado em Geofísica} (72 meses). Universidade de São Paulo.
  Orientadora: Naomi Ussami.
  ``Cálculo do tensor gradiente gravimétrico utilizando tesseroides.''
  \\
  \Duration{2010}{2011}  &
  \textbf{Mestrado em Geofísica} (20 meses). Observatório Nacional.
  Orientadora: Valéria C. F. Barbosa.
  ``Robust 3D gravity gradient inversion by planting anomalous densities.''
  \\
  \Duration{2011}{2016}  &
  \textbf{Doutorado em Geofísica} (54 meses). Observatório Nacional.
  Orientadora: Valéria C. F. Barbosa.
  ``Forward modeling and inversion of gravitational fields in spherical coordinates.''
  \\
  \Duration{2017}{2019}  &
  \textbf{Pós-doutorado} (30 meses). University of Hawai`i at M\={a}noa, E.U.A.
  Orientador: Paul Wessel.
  \\
  \Duration{2008}{2009}  &
  \textbf{Intercâmbio Internacional} (10 meses). York University, Canadá.
\end{EntriesTableDuration}

\section{Histórico profissional, serviços e distinções acadêmicas e prêmios}
% Listar as principais posições profissionais que ocupou informando datas de
% início, término, e instituições (essas posições podem ser acadêmicas,
% empresariais ou administrativas, como a gestão de grandes projetos ou de
% instituições de ensino e pesquisa). Podem também ser listadas atividades
% associativas (participação em associações, federações, comissões
% temáticas/técnicas, conselhos de empresas/institutos/universidades), bem como
% atuação em empreendedorismo e startups e distinções acadêmicas e prêmios
% recebidos.

\begin{EntriesTableDuration}
  \Duration{02/2014}{02/2018}  &
  \textbf{Professor Assistente}. Universidade do Estado do Rio de Janeiro.
  \newline
  Departamento de Geologia Aplicada -- Faculdade de Geologia.
  \\
  \Duration{08/2019}{07/2023}  &
  \textbf{Lecturer}. University of Liverpool, Reino Unido.
  \newline
  Department of Earth, Ocean and Ecological Sciences -- School of Environmental Sciences.
  \\
  \Duration{08/2023}{atual}  &
  \textbf{Professor Doutor}. Universidade de São Paulo.
  \newline
  Departamento de Geofísica -- Instituto de Astronomia, Geofísica e Ciências Atmosféricas.
  \\
  \Duration{06/2022}{atual}  &
  \textbf{Board Member}. Software Underground.
  \newline
  Departamento de Geofísica -- Instituto de Astronomia, Geofísica e Ciências Atmosféricas.
\end{EntriesTableDuration}

\section{Resultados de pesquisa mais relevantes e de maior impacto}

\section{Financiamentos à pesquisa}

\section{Indicadores quantitativos}

\section{Links}

\section{Outras informações}


\end{document}
%------------------------------------------------------------------------------


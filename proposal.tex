% Main proposal text for FAPESP.
%
%%%%%%%%%%%%%%%%%%%%%%%%%%%%%%%%%%%%%%%%%%%%%%%%%%%%%%%%%%%%%%%%%%%%%%%%%%%%%%%
% Set a class and general configuration
\documentclass[onecolumn,a4paper,12pt]{article}

%%%%%%%%%%%%%%%%%%%%%%%%%%%%%%%%%%%%%%%%%%%%%%%%%%%%%%%%%%%%%%%%%%%%%%%%%%%%%%%
% Set variables with the title, authors, etc.
% Edit the values and they will propagate to the document and head/foot

% Common variables for other documents defined in information.tex
\newcommand{\Title}{%
  Grant proposal title in English
}
\newcommand{\TitlePt}{%
  Grant proposal title in Portuguese
}
\newcommand{\TitleShort}{Short title for the footer}
\newcommand{\Year}{2024}
\newcommand{\TeamShort}{Uieda \textit{et al.}}
\newcommand{\PIname}{Leonardo Uieda}
\newcommand{\PIORCID}{0000-0001-6123-9515}
\newcommand{\Institution}{%
  Instituto de Astronomia, Geofísica e Ciências Atmosféricas -- Universidade de
  São Paulo
}
\newcommand{\FundingCall}{FAPESP Jovem Pesquisador}


\newcommand{\DocumentTitle}{\Title}
\newcommand{\Team}{%
  Person A\textsuperscript{1},
  Person B\textsuperscript{2}
}
\newcommand{\Affiliations}{%
  \textsuperscript{1}Observatório Nacional, Brasil;
  \textsuperscript{2}University of Liverpool, UK;
}

%%%%%%%%%%%%%%%%%%%%%%%%%%%%%%%%%%%%%%%%%%%%%%%%%%%%%%%%%%%%%%%%%%%%%%%%%%%%%%%
% Import the required packages
\usepackage[utf8]{inputenc}
\usepackage[TU]{fontenc}
\usepackage[english]{babel}
\usepackage{amsmath}
\usepackage{amssymb}
\usepackage{graphicx}
\usepackage{hyperref}
\usepackage{fancyhdr}
\usepackage{geometry}
\usepackage{booktabs}
\usepackage{microtype}
\usepackage{siunitx}
\usepackage{xcolor}
% To customize the title page
\usepackage{titling}
% improved urls with proper hyphenation
\usepackage{xurl}
% Tweak the look of captions
\usepackage{caption}
% To control the style of section titles
\usepackage{titlesec}
% Import natbib and doi packages
\usepackage[round,authoryear,sort]{natbib}
% show dois as links on references
\usepackage{doi}
% Remove extra space between references
\usepackage{natbibspacing}
% Use a different font
\usepackage[scaled=0.9,sfdefault]{notomath}
% Icons and fonts (requires using xelatex or luatex)
\usepackage{fontawesome5}
\usepackage{academicons}
% Control the font size
\usepackage{anyfontsize}
\usepackage{setspace}
% To get the number of pages in the document
\usepackage{lastpage}
\usepackage{lipsum}
\usepackage{ragged2e}
\usepackage{mdframed}
% To define custom environments
\usepackage{environ}
% To control hyphenation for individual blocks of text
\usepackage{hyphenat}
% To add line numbers
\usepackage{reledmac}

%%%%%%%%%%%%%%%%%%%%%%%%%%%%%%%%%%%%%%%%%%%%%%%%%%%%%%%%%%%%%%%%%%%%%%%%%%%%%%%
% Configuration of the document
\geometry{%
  left=30mm,
  right=15mm,
  top=15mm,
  bottom=15mm,
  headsep=0mm,
  headheight=0mm,
  footskip=7mm,
  includehead=true,
  includefoot=true
}

% Control line and table row spacing
\onehalfspacing
\renewcommand{\arraystretch}{1.5}

% Set the spacing between bibliography entries (requires natbib)
\setlength{\bibsep}{0pt}

% Custom colors
\definecolor{darkgray}{gray}{0.4}
\definecolor{mediumgray}{gray}{0.5}
\definecolor{lightgray}{gray}{0.9}
\definecolor{mediumblue}{HTML}{2060c2}
\definecolor{lightblue}{HTML}{f7faff}

% Configure captions
\captionsetup[table]{position=below,skip=0pt}
\captionsetup{labelfont=bf,font={small,color=darkgray},skip=10pt}

% Make urls use the same font as every other text
\urlstyle{same}

% Number every single line instead of every 5th line
\firstlinenum{1}
\linenumincrement{1}

% Configure hyperref and add PDF metadata
\hypersetup{
    colorlinks,
    allcolors=mediumblue,
    pdftitle={\DocumentTitle},
    pdfauthor={\PIname},
    breaklinks=true,
}

% Configure header and footer
% Inspired by LaPreprint: https://github.com/roaldarbol/LaPreprint
\newcommand{\Separator}{\hspace{3pt}|\hspace{3pt}}
\newcommand{\FooterFont}{\footnotesize\color{mediumgray}}
\pagestyle{fancy}
\fancyhf{}
\lfoot{%
  \FooterFont{}
  \TeamShort{} (\Year)
  \Separator{}
  \TitleShort
}
\rfoot{%
  \FooterFont{}
  \FundingCall{}
  \Separator{}
  \thepage\space of\space \pageref*{LastPage}
}
\renewcommand{\headrulewidth}{0pt}
\renewcommand{\footrulewidth}{1pt}
\preto{\footrule}{\color{lightgray}}
\fancypagestyle{plain}{%
  \fancyhf{}
  \lfoot{}
  \rfoot{}
  \renewcommand{\footrulewidth}{0pt}
}

% Define fancy text boxes
\NewEnviron{summarybox}{%
  \mdfdefinestyle{summarybox_}{%
    leftline=true,
    rightline=false,
    topline=false,
    bottomline=false,
    linewidth=2pt,
    linecolor=mediumblue,
    backgroundcolor=lightblue,
    innertopmargin=12pt,
    innerbottommargin=12pt,
    innerleftmargin=12pt,
    innerrightmargin=12pt,
    skipbelow=5pt,
    skipabove=5pt,
  }
  \newmdenv[style=summarybox_]{summarybox_}
  \begin{summarybox_}
    \footnotesize
    \BODY
  \end{summarybox_}
}



\begin{document}


%%%%%%%%%%%%%%%%%%%%%%%%%%%%%%%%%%%%%%%%%%%%%%%%%%%%%%%%%%%%%%%%%%%%%%%%%%%%%%%
% Title page in English
\thispagestyle{plain}
\begin{FlushLeft}
  \begin{spacing}{2}
    {\LARGE \textbf{\Title}}
  \end{spacing}
  \vspace{0.1cm}
  \textbf{Principle Investigator:} \PIname
  \\[0.3cm]
  \textbf{Host Institution:} \Institution
  \\[0.3cm]
  \textbf{Project Team:} \Team
  \\[0.1cm]
  {\footnotesize \Affiliations}
  \\[0.5cm]
  {\color{lightgray}\hrule height 1.5pt}
\end{FlushLeft}

\section*{\normalsize Summary}
\beginnumbering
\autopar


% Project summary in English (max 20 lines)
\lipsum[1-3]  % Delete this and write the summary here


\endnumbering
\vfill
\begin{center}
  \includegraphics[height=0.8cm]{figures/usp.png}\hspace{1cm}
  \includegraphics[height=0.8cm]{figures/iag.png}\hspace{1cm}
  \includegraphics[height=0.8cm]{figures/compgeolab.png}
\end{center}

%%%%%%%%%%%%%%%%%%%%%%%%%%%%%%%%%%%%%%%%%%%%%%%%%%%%%%%%%%%%%%%%%%%%%%%%%%%%%%%
% Title page in Portuguese
\newpage
\thispagestyle{plain}
\begin{FlushLeft}
  \begin{spacing}{2}
    {\LARGE \textbf{\TitlePt}}
  \end{spacing}
  \vspace{0.1cm}
  \textbf{Pesquisador Responsável:} \PIname
  \\[0.3cm]
  \textbf{Instituição Sede:} \Institution
  \\[0.3cm]
  \textbf{Equipe do Projeto:} \Team
  \\[0.1cm]
  {\footnotesize \Affiliations}
  \\[0.5cm]
  {\color{lightgray}\hrule height 1.5pt}
\end{FlushLeft}

\section*{\normalsize Resumo}
\beginnumbering
\autopar


% Project summary in Portuguese (max 20 lines)
\lipsum[1-3]  % Delete this and write the summary here


\endnumbering
\vfill
\begin{center}
  \includegraphics[height=0.8cm]{figures/usp.png}\hspace{1cm}
  \includegraphics[height=0.8cm]{figures/iag.png}\hspace{1cm}
  \includegraphics[height=0.8cm]{figures/compgeolab.png}
\end{center}

\newpage
\setcounter{page}{1}

%%%%%%%%%%%%%%%%%%%%%%%%%%%%%%%%%%%%%%%%%%%%%%%%%%%%%%%%%%%%%%%%%%%%%%%%%%%%%%%
\section{Problem statement}
% What is the problem being addressed and what is it's importance? What will be
% the contribution to the field if this works?



\lipsum[1-5]



%%%%%%%%%%%%%%%%%%%%%%%%%%%%%%%%%%%%%%%%%%%%%%%%%%%%%%%%%%%%%%%%%%%%%%%%%%%%%%%
\section{Expected results}
% What will be created/produced as a result?

\lipsum[1-5]





%%%%%%%%%%%%%%%%%%%%%%%%%%%%%%%%%%%%%%%%%%%%%%%%%%%%%%%%%%%%%%%%%%%%%%%%%%%%%%%
\section{Scientific and technological challenges}
% Main challenges faced and how they will be overcome. Explain the main
% scientific/technological challenges that will need to be overcome to achieve
% the project goals. Describe methods that will be employed. Cite literature
% that will allow referees to determine that the challenges were not already
% overcome and that the methods employed are appropriate.

\lipsum[1-5]


%%%%%%%%%%%%%%%%%%%%%%%%%%%%%%%%%%%%%%%%%%%%%%%%%%%%%%%%%%%%%%%%%%%%%%%%%%%%%%%
\section{Schedule}
% When will the project stages be completed? Which events/landmarks can be used
% to measure progress and completion?


\lipsum[1-5]


%%%%%%%%%%%%%%%%%%%%%%%%%%%%%%%%%%%%%%%%%%%%%%%%%%%%%%%%%%%%%%%%%%%%%%%%%%%%%%%
\section{Dissemination and evaluation}
% How will the project results be evaluated and disseminated to the public and
% the scientific community?


\lipsum[1-5]


%%%%%%%%%%%%%%%%%%%%%%%%%%%%%%%%%%%%%%%%%%%%%%%%%%%%%%%%%%%%%%%%%%%%%%%%%%%%%%%
\section{Other support}
% Indique outros apoios ao projeto, se houver, em forma de fundos, bens ou
% serviços, mas sem incluir itens como uso de instalações da instituição que já
% estão disponíveis. Note que os autores das propostas selecionadas deverão
% apresentar documento assinado pelo dirigente da Instituição Sede,
% comprometendo os recursos e bens adicionais descritos na proposta.



\lipsum[1-5]



%%%%%%%%%%%%%%%%%%%%%%%%%%%%%%%%%%%%%%%%%%%%%%%%%%%%%%%%%%%%%%%%%%%%%%%%%%%%%%%
% Bibliography
\newpage
\bibliographystyle{apalike-doi}
\bibliography{references}

\end{document}
%------------------------------------------------------------------------------
